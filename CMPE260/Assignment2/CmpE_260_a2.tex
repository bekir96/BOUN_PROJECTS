\documentclass{article}
\usepackage[utf8]{inputenc}
\usepackage{listings,tcolorbox}
\title{CmpE 260 - Principles of Programming Languages \\Spring 2019 \\Assignment 2}
\author{Bekir Yıldırım - 2014400054}
\date{due: 27.05.2019 - 11:59}

\usepackage{natbib}
\usepackage{graphicx}
\usepackage{amsmath}
\usepackage{enumitem}
\usepackage{float}
\usepackage{hyperref}
\usepackage[margin=1.25in]{geometry}
\usepackage{textcomp}
\hypersetup{
    colorlinks=true,
    linkcolor=blue,
    filecolor=magenta,      
    urlcolor=cyan,
}

\def\signed #1{{\leavevmode\unskip\nobreak\hfil\penalty50\hskip2em
  \hbox{}\nobreak\hfil(#1)%
  \parfillskip=0pt \finalhyphendemerits=0 \endgraf}}

\newsavebox\mybox
\newenvironment{aquote}[1]
  {\savebox\mybox{#1}\begin{quote}}
  {\signed{\usebox\mybox}\end{quote}}
  
\begin{document}

\maketitle

\section*{Answer 1}
(Code lines are designated with color \textcolor{red}{red})
\\(Print statement with color \textcolor{blue}{blue} )
\lstset{
                language=C++,
                basicstyle=\ttfamily,
                keywordstyle=\color{blue}\ttfamily,
                stringstyle=\color{blue}\ttfamily,
                commentstyle=\color{red}\ttfamily,
                morecomment=[l][\color{magenta}]{\#},
                frame=single
}
\begin{lstlisting}
int n = 50; // global           //1
print_plus_n(int x) {           //2
   cout << x + n << endl;       //3
}                               //4
increment_n() {                 //5
   n = n + 1;                   //6
}                               //7
main() {                        //8
   int n;                       //9
   n = 10;                      //10
   print_plus_n(n);             //11
   increment_n();               //12
   cout << n << endl;           //13
   print_plus_n(n);             //14
}                               //15
\end{lstlisting} \\~\\ 
\begin{enumerate}[label=\textbf{\alph*)}]
\item \textbf{Static Scoping} 
    
    \textit{Static Scoping} is a convention used with many programming languages that sets the scope (range of functionality) of a variable so that it may only be called (referenced) from within the block of code in which it is defined. The scope is determined when the code is compiled. 
    \\~\\ \textbf{Explanation}  
    \begin{enumerate}[label=\textbf{\alph*)}]
    \item \textbf{First Output} \\~\\
    Main function is the entry point of this C++ program. An integer variable n is declared in \textcolor{red}{(9)} and assigned a value of 10 in \textcolor{red}{(10)}. print\_plus\_n(n) definition is calling with a actual parameter of n for 10 and print\_plus\_n(int x) declaration in \textcolor{red}{(2)} with formal parameter, x, is called and value of actual parameter, 10, is passed to formal parameter, x.
    x is assigned with 10 but no declaration is found for the variable n, so search continues in the declarations of the subprogram that declared subprogram print\_plus\_n(int x), which it is called its static parent. And a declaration for  n is found in \textcolor{red}{(1)}. Therefore, x+n becomes number of 60 so cout prints \textcolor{blue}{60}.\\
    
    \item \textbf{Second Output} 
    \\~\\ increment\_n() statement is defined in main \textcolor{red}{(12)} and declaration of this function at \textcolor{red}{(5)}. This function have no actual and formal parameters so n has no declaration there. To find an assigned value, the search continues and a declaration for  n is found in \textcolor{red}{(1)}. value of n becomes 51 in \textcolor{red}{(6)} but this value cannot pass the line \textcolor{red}{(13)} since in static scoping, if a variable is static variable, it changes just in that subprogram. So cout in \textcolor{red}{(13)} prints \textcolor{blue}{10}.\\
   
    \item \textbf{Third Output} \\~\\
    print\_plus\_n(n) function is called from main like as First Output, just there is difference in value of n. Because, \textcolor{red}{(23)} sets global value of n to 51, moreover third output is \newline 10+51 = 61 and \textcolor{red}{(3)} prints \textcolor{blue}{61}
    \end{enumerate}
    \\~\\ Outputs are given below respectively: \\ \textcolor{blue}{60} \\ \textcolor{blue}{10} \\ \textcolor{blue}{61} \\~\\
    
\item \textbf{Dynamic scoping} 

\textit{Dynamic Scoping} does not care how the code is written, but instead how it executes. Each time a new function is executed, a new scope is pushed onto the stack. This scope is typically stored with the function’s call stack. When a variable is referenced in the function, the scope in each call stack is checked to see if it provides the value. \\~\\
\textbf{Explanation}  
\begin{enumerate}[label=\textbf{\alph*)}]
    \item \textbf{First Output} \\~\\
Main function is the entry point of this C++ program. An integer variable n is declared in \textcolor{red}{(9)} and assigned a value of 10 in \textcolor{red}{(10)}. print\_plus\_n(n) definition is calling with a actual parameter of n for 10 and print\_plus\_n(int x) declaration in \textcolor{red}{(2)} with formal parameter, x, is called and value of actual parameter, 10, is passed to formal parameter, x. In dynamic programming, global n becomes 10 because of assigned value of n in \textcolor{red}{(10)}.
    x is assigned with 10 but no declaration is found for the variable n, so search continues in the declarations of the subprogram that declared subprogram print\_plus\_n(int x). And a declaration for  n is found in \textcolor{red}{(1)}. Therefore, x+n becomes number of 20 so cout prints \textcolor{blue}{20}\\
   
    
    \item \textbf{Second Output} \\~\\
    increment\_n() statement is defined in main \textcolor{red}{(12)} and declaration of this function at \textcolor{red}{(5)}. This function have no actual and formal parameters so n has no declaration there. To find an assigned value, the search continues and a declaration for  n is found with 10 in \textcolor{red}{(1)}. value of n becomes 11 in \textcolor{red}{(6)} and this value can pass the line \textcolor{red}{(13)} since in dynamic scoping, if a variable is defined and assigned at any time, it changes in everywhere. So cout in \textcolor{red}{(13)} prints \textcolor{blue}{(11)}.
    
    \item \textbf{Third Output} 
    \\~\\ 
    print\_plus\_n(n) function is called from main like as First Output, just there is difference in value of n. Because, \textcolor{red}{(23)} sets global value of n to 11, moreover third output is \newline 11+11 = 22 and \textcolor{red}{(3)} prints \textcolor{blue}{22}
    \end{enumerate}
    \\~\\ Outputs are given below respectively: \\ \textcolor{blue}{20} \\ \textcolor{blue}{11} \\ \textcolor{blue}{22} \\~\\
\end{enumerate}


\section*{Answer 2}

\begin{lstlisting}
void function1(int a, int b, int c) {           //1
   int d = -1;                                  //2
   while(a > 0) {                               //3
      c = c / b;                                //4
      a = a + d;                                //5
    }                                           //6
}                                               //7
void function2(int a, int b) {                  //8
   int c = b;                                   //9
   b = a;                                       //10
   a = c;                                       //11
}                                               //12
int main() {                                    //13
   int x = 2;                                   //14
   int y = 10;                                  //15
   int z = 1500;                                //16
   cout << x << "," << y << "," << z << endl;   //17
   function1(x,y,z);                            //18
   cout << x << "," << y << "," << z << endl;   //19
   function2(x,z)                               //20
   cout << x << "," << y << "," << z << endl;   //21
}                                               //22
\end{lstlisting}
\\~\\ 
\begin{enumerate}[label=\textbf{\alph*)}]
    \item \textbf{Pass-by-value}\\

Main function is the entry point of this C++ program and x,y,z variables are initialized by 2,10 and 1500 respectively. cout at \textcolor{red}{(17)} prints  \textcolor{blue}{2,10,1500} \textcolor{red}{(17)}. \\ Later, 
function1(x,y,z) definition \textcolor{red}{(18)} is calling its declaration in \textcolor{red}{(1)} and x,y,z actual parameters which have values 2,10,1500 respectively are sent to formal parameters a,b and c respectively. But in function1, not see any statement about ability to change local variables in main function. So, 
\textcolor{red}{(19)} prints 
\textcolor{blue}{2,10,1500}.\\
At last, function2(x,z) is similar as function1 in other words, function2 has not also any statement about ability to change local variables in main function. So, 
\textcolor{red}{(21)} prints 
\textcolor{blue}{2,10,1500}.
\\~\\Outputs are given below respectively: \\ \textcolor{blue}{2,10,1500\\2,10,1500\\2,10,1500}  \\~\\ 

    \item \textbf{Pass-by-reference} 

Main function is the entry point of this C++ program and x,y,z variables are initialized by 2,10 and 1500 respectively. cout at \textcolor{red}{(17)} prints  \textcolor{blue}{2,10,1500} \textcolor{red}{(17)}.\\
Later, function1(x,y,z) definition \textcolor{red}{(18)} is calling its declaration in \textcolor{red}{(1)} and x,y,z actual parameters which have values 2,10,1500 respectively are sent to formal parameters a,b and c respectively.
Local variable d is initialized with value '-1'.
In pass by reference, any change to formal parameters directly affects x,y and z. In the while loop, until a is 0, it will continue subtracting 1 from a and dividing c by b. loop will turn 2 times.
When the loop exits, a, b and c take values 0, 10 and 15 respectively, so by the pass-by-reference, x, y and z values have changed.
cout prints \textcolor{blue}{0,10,15} \textcolor{red}{(19)} \\
At last, function2(x,z) is called with values actual parameters x and z, (0,15), and a and b take them. 
In the function2 body, a local variable c is initialized with value 15 from b. b takes value 0 from a, finally a takes value 15 from c. So, the last values of x,y and z are 15, 10 and 0 respectively.
cout prints \textcolor{blue}{15,10,0} \textcolor{red}{(21)} \\~\\
Outputs are given below respectively: \\ \textcolor{blue}{2,10,1500\\0,10,15\\15,10,0} \\~\\ 
    \item \textbf{Pass-by-name} \\
Main function is the entry point of this C++ program and x,y,z variables are initialized by 2,10 and 1500 respectively. cout at \textcolor{red}{(17)} prints  \textcolor{blue}{2,10,1500} \textcolor{red}{(17)}.\\
Later, function1(x,y,z) definition \textcolor{red}{(18)} is calling its declaration in \textcolor{red}{(1)} and x,y,z actual parameters which have values 2,10,1500 respectively are sent to formal parameters a,b and c respectively.
Local variable d is initialized with value '-1'.
In pass by name, any change to formal parameters directly affects x,y and z since pass-by-name in an inout mode parameter transmission method that does not correspond to a single implementation. In the while loop, until a is 0, it will continue subtracting 1 from a and dividing c by b. loop will turn 2 times.
When the loop exits, a, b and c take values 0, 10 and 15 respectively, so by the pass-by-name, x, y and z values have changed.
cout prints \textcolor{blue}{0,10,15} \textcolor{red}{(19)} \\
At last, function2(x,z) is called with values actual parameters x and z, (0,15), and a and b take them. 
In the function2 body, a local variable c is initialized with value 15 from b. b takes value 0 from a, finally a takes value 15 from c. So, the last values of x,y and z are 15, 10 and 0 respectively.
cout prints \textcolor{blue}{15,10,0} \textcolor{red}{(21)} \newpage
Outputs are given below respectively: \\ \textcolor{blue}{2,10,1500\\0,10,15\\15,10,0}
\end{enumerate}

\end{document}